\documentclass{article}

\usepackage{natbib}
\usepackage{graphicx}
\usepackage{amsmath}
\usepackage{lipsum}

\title{A Detailed Characterization of Heat Flows in the Ground Under a Building with Optimization for Simulation} 
\author{
  Mitchell, Matt\\
  \texttt{matt.mitchell@okstate.edu}
  \and
  Lee, Edwin\\
  \texttt{edwin.lee@nrel.gov}
}
\date{}

\begin{document}
 
 \maketitle
 
 \section*{Abstract}
  \lipsum[1]

 \section*{Introduction}
  Provide the problem statement
  \lipsum[1]

  \subsection*{Background}
   A more detailed introduction with background and literature review.  Interesting citations: Kiva, Slab/Basement, HVAC\&R paper \citep{Lee2013}, etc.  By the way, Figure~\ref{fig:google} shows a picture of the Google Logo.
   \begin{figure}
    \centering
    \label{fig:google}
    \includegraphics[width=0.7\textwidth]{images/googlelogo.png}
    \caption{Google Logo}
   \end{figure}
   \lipsum[1]

  \subsection*{Scope and Purpose}
   What are the actual goals of this paper?  By the way, Figure~-ref{fig:input-example} shows a plot that was generated by the build system.
   \begin{figure}
    \centering
    \label{fig:input-example}
    \includegraphics[width=0.7\textwidth]{images-generated/input_example.png}
    \caption{Generated Plot}
   \end{figure}
   \lipsum[1]

  \subsection*{Mathematical Approach}
   What is the mathematical reasoning for the approach used here...how does taking a temperature slice in the domain help?  By the way, Equation~\eqref{eq:einstein} is a classic equation.
   \begin{equation}
    \label{eq:einstein}
    E = m c^2
   \end{equation}
   \lipsum[1]

 \section*{Preliminary Model Evaluation} 
  Describe the model, focus on three phases: initial implementation (cite the old paper), experimental and other validation to build confidence, and then changes since then.  Description of BESTEST results here.
  \lipsum[1]
 
 \section*{Experiment A: Characterizing Heat Flows}
  \lipsum[1]

  \subsection*{Ground/Building configuration and description}
   What are we modeling?

  \subsection*{Boundary conditions and ground properties}
   Yep. 

  \subsection*{Specific slices}
   Describle the slices we'll be taking through the domain 

  \subsection*{Results}
   How did it turn out?  Lots of pictures.

 \section*{Experiment B: Optimizing for RunTime}
  \lipsum[1]

  \subsection*{Tolerances and Approach}
   How will we proceed forward, ensuring accuracy is maintained while reducing runtime

  \subsection*{Information from Experiment A}
   Wrap the results of Experiment A into the expected target regions of the domain

  \subsection*{Results}
   How did we do in each test?

 \section*{Conclusions}
  We did awesome.  Here are some additional things that would be cool
  \lipsum[1]

 \bibliographystyle{plainnat}
 \bibliography{bibfile} 

\end{document}
