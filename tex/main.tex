\documentclass{article}

\usepackage{natbib}

\title{A Detailed Characterization of Heat Flows in the Ground Under a Building with Optimization for Simulation} 
\author{
  Mitchell, Matt\\
  \texttt{matt.mitchell@something.com}
  \and
  Lee, Edwin\\
  \texttt{edwin.lee@nrel.gov}
}
\date{}

\begin{document}
 
 \maketitle
 
 \section*{Abstract}
 
 \section*{Introduction}
     Provide the problem statement
  \subsection*{Background}
     A more detailed introduction with background and literature review.  Interesting citations: Kiva, Slab/Basement, HVAC\&R paper \citep{Lee2013}, etc.
  \subsection*{Scope and Purpose}
     What are the actual goals of this paper?
  \subsection*{Mathematical reasoning for the approach used here...how does taking a temperature slice in the domain help?}
 
 \section*{Preliminary Model Evaluation} 
  Describe the model, focus on three phases: initial implementation (cite the old paper), experimental and other validation to build confidence, and then changes since then.  Description of BESTEST results here.
 
 \section*{Experiment A: Characterizing Heat Flows}
  \subsection*{Ground/Building configuration and description}
   What are we modeling?
  \subsection*{Boundary conditions and ground properties}
   Yep. 
  \subsection*{Specific slices}
   Describle the slices we'll be taking through the domain 
  \subsection*{Results}
   How did it turn out?  Lots of pictures.

 \section*{Experiment B: Optimizing for RunTime}
  \subsection*{Tolerances and Approach}
   How will we proceed forward, ensuring accuracy is maintained while reducing runtime
  \subsection*{Information from Experiment A}
   Wrap the results of Experiment A into the expected target regions of the domain
  \subsection*{Results}
   How did we do in each test?

 \section*{Conclusions}
  We did awesome.  Here are some additional things that would be cool

 \section*{References}
  \bibliographystyle{plainnat}
  \bibliography{bibfile} 

\end{document}
